
      \documentclass[a4paper,fleqn,papersize,15pt]{jsarticle}
      \begin{document}
      ichimura

 (1,3)  \large     $3.6 + x = -1.4$

(2,6)  \large     $- \frac{9}{8}x -4 = -3x$

(2,7)  \large     $\frac{x-5}{4} + 6 = \frac{5}{2}$

(2,8)  \large     $\frac{2x+1}{5}  = \frac{x-5}{4}$

(3,10)  \large     一個130円のケーキを何個か買い,60円の箱に詰めてもらったところ代金の合計は1100円であった.ケーキを何個買ったか.

(4,6)  \large     $\frac{7}{3} -7x = -3x$

(4,7)  \large     $- \frac{2x-1}{9} + 2 = \frac{5}{2}$

(4,8)  \large     $\frac{3}{4}x-\frac{1}{2}x = \frac{2}{3}-\frac{7}{6}x$

(4,9)  \large     $ 3 - \frac{x-3}{6} = \frac{2x-3}{3}$

\clearpage
isii    

 (2,3)  \large     $7.2x -4 = 2.2x + 9$

(2,4)  \large     $3(x -4) = 7x -10$

(2,7)  \large     $\frac{x-5}{4} + 6 = \frac{5}{2}$

(2,8)  \large     $\frac{2x+1}{5}  = \frac{x-5}{4}$

(2,9)  \large     $0.8x-(0.01x-2) = 0.03x-2$

(2,10)  \large     連続する三つの整数があり,その和は45である.この三つの整数を求めよ.

(3,4)  \large     $9(x -2) = 2x + 1$

(3,9)  \large     $\frac{x+3}{5} +0.5x = -\frac{1}{10}$

(5,3)  \large     $4(3x + 1) = 4x -6$

(5,4)  \large     $3x -6 = -9x + 10$

(6,9)  \large     $\frac{2}{7}x - \frac{2}{3} = -7$

(6,10)  \large     xについての方程式$3(4x-a)+2(x-2a)=0$の解が2であるとき,aの値を求めよ.

(7,9)  \large     $\frac{5}{3}x - \frac{3}{2}x = 6$

\clearpage
ootuki  

 (1,3)  \large     $3.6 + x = -1.4$

(1,7)  \large     $\frac{3}{7}x + 2 = \frac{6}{7}$

(1,9)  \large     $\frac{x+1}{3} +\frac{2x+1}{2} = \frac{3x-4}{2}$

(2,3)  \large     $7.2x -4 = 2.2x + 9$

(2,4)  \large     $3(x -4) = 7x -10$

(2,5)  \large     $\frac{3}{7}x -3 = -5$

(2,6)  \large     $- \frac{9}{8}x -4 = -3x$

(2,7)  \large     $\frac{x-5}{4} + 6 = \frac{5}{2}$

(2,8)  \large     $\frac{2x+1}{5}  = \frac{x-5}{4}$

(2,9)  \large     $0.8x-(0.01x-2) = 0.03x-2$

(3,3)  \large     $3x + 5 = -x + 2$

(3,6)  \large     $\frac{3}{2}x -6 = -5x$

(3,7)  \large     $\frac{5}{2}x + 2 = 2x$

(3,9)  \large     $\frac{x+3}{5} +0.5x = -\frac{1}{10}$

(4,5)  \large     $\frac{7}{3}x + 2 = 1$

(4,6)  \large     $\frac{7}{3} -7x = -3x$

(4,7)  \large     $- \frac{2x-1}{9} + 2 = \frac{5}{2}$

(4,8)  \large     $\frac{3}{4}x-\frac{1}{2}x = \frac{2}{3}-\frac{7}{6}x$

(4,9)  \large     $ 3 - \frac{x-3}{6} = \frac{2x-3}{3}$

(4,10)  \large     一個40円のリンゴと80円のミカンを合わせて15個買ったら,代金の合計は840円であった.ミカンは何個買いましたか.

(5,8)  \large     $\frac{1}{3}x + 0.5x  = \frac{10}{9}$

(6,3)  \large     $7(x + 3) = -5x -8$

(6,8)  \large     $\frac{7}{4} + 8x = - \frac{4}{3}$

(6,9)  \large     $\frac{2}{7}x - \frac{2}{3} = -7$

(7,7)  \large     $\frac{x-2}{3}  = \frac{8}{7}$

(7,8)  \large     $\frac{4}{7}x -0.2x = - \frac{3}{8}$

\clearpage
ono     

 (1,2)  \large     $5x - 7 = 4$

(2,9)  \large     $0.8x-(0.01x-2) = 0.03x-2$

(4,9)  \large     $ 3 - \frac{x-3}{6} = \frac{2x-3}{3}$

(5,3)  \large     $4(3x + 1) = 4x -6$

(5,5)  \large     $\frac{10}{9}x -3 = -5$

(5,8)  \large     $\frac{1}{3}x + 0.5x  = \frac{10}{9}$

(5,9)  \large     $-0.4( \frac{5}{2}x+12 ) = -3$

(6,3)  \large     $7(x + 3) = -5x -8$

(6,5)  \large     $\frac{8}{5}x -4 = -7$

(6,7)  \large     $- \frac{x+3}{3} -3 = - \frac{5}{3}$

(6,10)  \large     xについての方程式$3(4x-a)+2(x-2a)=0$の解が2であるとき,aの値を求めよ.

(7,3)  \large     $-2(x + 8) = 6x - 6$

(7,7)  \large     $\frac{x-2}{3}  = \frac{8}{7}$

(7,8)  \large     $\frac{4}{7}x -0.2x = - \frac{3}{8}$

(7,9)  \large     $\frac{5}{3}x - \frac{3}{2}x = 6$

(7,10)  \large     現在,父は42歳で子は12歳である.父の年齢が子供の年齢のちょうど2倍になるのは今から何年後か

\clearpage
okano   

 (1,8)  \large     $- \frac{4}{7}x + 3 = - \frac{1}{7}x$

(3,4)  \large     $9(x -2) = 2x + 1$

(3,9)  \large     $\frac{x+3}{5} +0.5x = -\frac{1}{10}$

(6,3)  \large     $7(x + 3) = -5x -8$

(6,7)  \large     $- \frac{x+3}{3} -3 = - \frac{5}{3}$

(6,9)  \large     $\frac{2}{7}x - \frac{2}{3} = -7$

(7,4)  \large     $-6x + 1 = 4x -7$

(7,9)  \large     $\frac{5}{3}x - \frac{3}{2}x = 6$

(8,3)  \large     $-5(x -1) = -7(x + 3)$

(8,6)  \large     $\frac{7}{4} + 3x = -1$

\clearpage
kuria   

 (3,6)  \large     $\frac{3}{2}x -6 = -5x$

(3,8)  \large     $\frac{3x-1}{2} + 1 = \frac{1}{3}$

(3,9)  \large     $\frac{x+3}{5} +0.5x = -\frac{1}{10}$

(4,7)  \large     $- \frac{2x-1}{9} + 2 = \frac{5}{2}$

(4,8)  \large     $\frac{3}{4}x-\frac{1}{2}x = \frac{2}{3}-\frac{7}{6}x$

(6,5)  \large     $\frac{8}{5}x -4 = -7$

(6,9)  \large     $\frac{2}{7}x - \frac{2}{3} = -7$

(7,4)  \large     $-6x + 1 = 4x -7$

(7,8)  \large     $\frac{4}{7}x -0.2x = - \frac{3}{8}$

(11,7)     \large     $\frac{3}{7}x + 2 = \frac{6}{7}$

(12,4)     \large     $\frac{3}{7}x -3 = -5$   \begin{flushright}\framebox[8em]{\rule{0pt}{6ex}}\end{flushright}

(12,8)     \large     $0.8x-(0.01x-2) = 0.03x-2$   \begin{flushright}\framebox[8em]{\rule{0pt}{6ex}}\end{flushright}

\clearpage
tukimori

 (1,10)  \large     ある数の5倍から2引いたら43になった.ある数を求めよ.

(2,3)  \large     $7.2x -4 = 2.2x + 9$

(2,4)  \large     $3(x -4) = 7x -10$

(2,6)  \large     $- \frac{9}{8}x -4 = -3x$

(2,7)  \large     $\frac{x-5}{4} + 6 = \frac{5}{2}$

(2,8)  \large     $\frac{2x+1}{5}  = \frac{x-5}{4}$

(2,9)  \large     $0.8x-(0.01x-2) = 0.03x-2$

(3,4)  \large     $9(x -2) = 2x + 1$

(3,6)  \large     $\frac{3}{2}x -6 = -5x$

(3,8)  \large     $\frac{3x-1}{2} + 1 = \frac{1}{3}$

(3,9)  \large     $\frac{x+3}{5} +0.5x = -\frac{1}{10}$

(3,10)  \large     一個130円のケーキを何個か買い,60円の箱に詰めてもらったところ代金の合計は1100円であった.ケーキを何個買ったか.

(4,7)  \large     $- \frac{2x-1}{9} + 2 = \frac{5}{2}$

(4,8)  \large     $\frac{3}{4}x-\frac{1}{2}x = \frac{2}{3}-\frac{7}{6}x$

(4,9)  \large     $ 3 - \frac{x-3}{6} = \frac{2x-3}{3}$

(5,2)  \large     $4x + 2 = 3$

(5,3)  \large     $4(3x + 1) = 4x -6$

(5,6)  \large     $\frac{5}{2}x -1 = x$

(5,7)  \large     $\frac{3x-2}{7} -2 = \frac{2}{7}$

(5,8)  \large     $\frac{1}{3}x + 0.5x  = \frac{10}{9}$

(6,7)  \large     $- \frac{x+3}{3} -3 = - \frac{5}{3}$

(6,9)  \large     $\frac{2}{7}x - \frac{2}{3} = -7$

(6,10)  \large     xについての方程式$3(4x-a)+2(x-2a)=0$の解が2であるとき,aの値を求めよ.

\clearpage
watanabe

 (1,9)  \large     $\frac{x+1}{3} +\frac{2x+1}{2} = \frac{3x-4}{2}$

(2,7)  \large     $\frac{x-5}{4} + 6 = \frac{5}{2}$

(2,8)  \large     $\frac{2x+1}{5}  = \frac{x-5}{4}$

(2,9)  \large     $0.8x-(0.01x-2) = 0.03x-2$

(2,10)  \large     連続する三つの整数があり,その和は45である.この三つの整数を求めよ.

(3,6)  \large     $\frac{3}{2}x -6 = -5x$

(3,10)  \large     一個130円のケーキを何個か買い,60円の箱に詰めてもらったところ代金の合計は1100円であった.ケーキを何個買ったか.

(4,1)  \large     $-10x - 9 = -5$

(4,10)  \large     一個40円のリンゴと80円のミカンを合わせて15個買ったら,代金の合計は840円であった.ミカンは何個買いましたか.

(6,3)  \large     $7(x + 3) = -5x -8$

(7,6)  \large     $- \frac{6}{7} + 4x = 2x$

(7,8)  \large     $\frac{4}{7}x -0.2x = - \frac{3}{8}$

(7,9)  \large     $\frac{5}{3}x - \frac{3}{2}x = 6$

\clearpage

        \end{document}
        \end{document}
        