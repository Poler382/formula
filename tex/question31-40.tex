\documentclass[a4paper,fleqn,papersize,15pt]{jsarticle}

\setlength{\topmargin}{-1in}
\addtolength{\topmargin}{5mm}
\setlength{\headheight}{5mm}
\setlength{\headsep}{0mm}

\setlength{\textheight}{\paperheight}
\addtolength{\textheight}{-25mm}
\setlength{\footskip}{5mm}
\renewcommand{\labelenumi}{(\theenumi)}
\begin{document}
 \begin{center}
   \LARGE\textbf{宮プリ ~一次方程式特訓編~\textcircled{\scriptsize 11}}
     \begin{flushright}
       名前\underline{\hspace{8zw}}
     \end{flushright}
 \end{center}

 \begin{itemize}
   \item 次の一次方程式をxについて解け
   \begin{enumerate}
\item $6x-10=2x$ \begin{flushright}\framebox[8em]{\rule{0pt}{6ex}}\end{flushright} %48
\item $0.6x+0.4(3x-2)=4.6$ \begin{flushright}\framebox[8em]{\rule{0pt}{6ex}}\end{flushright} %105
\item $-3x=-27$ \begin{flushright}\framebox[8em]{\rule{0pt}{6ex}}\end{flushright} %7
\item $\frac{1}{2} x+1= \frac{2}{3} x-5$ \begin{flushright}\framebox[8em]{\rule{0pt}{6ex}}\end{flushright} %115
\item $\frac{3}{4} x-1=2$ \begin{flushright}\framebox[8em]{\rule{0pt}{6ex}}\end{flushright} %154
\item $\frac{6}{5} x+ \frac{5}{2} =- \frac{3}{10} x-2$ \begin{flushright}\framebox[8em]{\rule{0pt}{6ex}}\end{flushright} %116
\item $x-4=7$ \begin{flushright}\framebox[8em]{\rule{0pt}{6ex}}\end{flushright} %1
\item 1200mの道のりを一定の速さで40分で歩いた。このときの速さは分速何mか。 \vfill \begin{flushright}\framebox[8em]{\rule{0pt}{6ex}}\end{flushright} %12
\item ある中学校の今年の生徒数は去年に比べて5\%増えて、441人でした。去年の生徒数を求めよ。 \vfill \begin{flushright}\framebox[8em]{\rule{0pt}{6ex}}\end{flushright} %41
\item 10\%の食塩水300gと1\%の食塩水を何gかをよく混ぜて、そこに食塩を20g入れ、さらにそこから水を70g蒸発させたら6\%の食塩水になった。1\%の食塩水は何gまぜたのだろうか。 \vfill \begin{flushright}\framebox[8em]{\rule{0pt}{6ex}}\end{flushright} %24
\end{enumerate}
    \vfill
\end{itemize}
\clearpage
 \begin{center}
   \LARGE\textbf{宮プリ ~一次方程式特訓編~\textcircled{\scriptsize 11}}
     \begin{flushright}
       名前\underline{\hspace{8zw}}
     \end{flushright}
 \end{center}

 \begin{itemize}
   \item 次の一次方程式をxについて解け
   \begin{enumerate}
\item $7+x=3$ \begin{flushright}\framebox[8em]{\rule{0pt}{6ex}}\end{flushright} %10
\item $\frac{3x-8}{6} =4x-3$ \begin{flushright}\framebox[8em]{\rule{0pt}{6ex}}\end{flushright} %144
\item $2x=1$ \begin{flushright}\framebox[8em]{\rule{0pt}{6ex}}\end{flushright} %12
\item $9-8x=11$ \begin{flushright}\framebox[8em]{\rule{0pt}{6ex}}\end{flushright} %32
\item $7x-8=4(4+x)$ \begin{flushright}\framebox[8em]{\rule{0pt}{6ex}}\end{flushright} %65
\item $2x=-12$ \begin{flushright}\framebox[8em]{\rule{0pt}{6ex}}\end{flushright} %6
\item $\frac{1}{2} x-4=11$ \begin{flushright}\framebox[8em]{\rule{0pt}{6ex}}\end{flushright} %109
\item 1200のt\%を文字式で表せ \vfill \begin{flushright}\framebox[8em]{\rule{0pt}{6ex}}\end{flushright} %34
\item 男子が18人、女子が22人のクラスがある。男子の平均点が70点、クラス全体の平均点は75.5点でした。女子の平均点を求めよ。 \vfill \begin{flushright}\framebox[8em]{\rule{0pt}{6ex}}\end{flushright} %31
\item ある中学校では全校生徒の45\%が女子である。男子の人数は女子の人数より24人多い。この学校の全校生徒の人数を求めよ。 \vfill \begin{flushright}\framebox[8em]{\rule{0pt}{6ex}}\end{flushright} %43
\end{enumerate}
    \vfill
\end{itemize}
\clearpage
 \begin{center}
   \LARGE\textbf{宮プリ ~一次方程式特訓編~\textcircled{\scriptsize 11}}
     \begin{flushright}
       名前\underline{\hspace{8zw}}
     \end{flushright}
 \end{center}

 \begin{itemize}
   \item 次の一次方程式をxについて解け
   \begin{enumerate}
\item $4x=-16$ \begin{flushright}\framebox[8em]{\rule{0pt}{6ex}}\end{flushright} %13
\item $\frac{7x+8}{12} = \frac{5x-9}{8}$ \begin{flushright}\framebox[8em]{\rule{0pt}{6ex}}\end{flushright} %148
\item $3x=15$ \begin{flushright}\framebox[8em]{\rule{0pt}{6ex}}\end{flushright} %4
\item $3(x-1)-2(x+3)=6-(2x+3)$ \begin{flushright}\framebox[8em]{\rule{0pt}{6ex}}\end{flushright} %163
\item $\frac{2}{3} (9-6x)-4(x+ \frac{1}{2} )=- \frac{3}{2} (8x-12)$ \begin{flushright}\framebox[8em]{\rule{0pt}{6ex}}\end{flushright} %88
\item $\frac{x}{6} = \frac{3-x}{4}$ \begin{flushright}\framebox[8em]{\rule{0pt}{6ex}}\end{flushright} %140
\item $2+3x=x+14$ \begin{flushright}\framebox[8em]{\rule{0pt}{6ex}}\end{flushright} %49
\item ある品物を仕入れて、仕入れ値の5割の利益を見込んで定価をつけた。定価では全く売れなかったので定価の3割引で売った。品物1個につき10円の利益になった。仕入れ値をx円として問いに答えよ。 \vfill \begin{flushright}\framebox[8em]{\rule{0pt}{6ex}}\end{flushright} %49
\begin{enumerate}
\item 定価をxを用いて表せ。 \vfill \begin{flushright}\framebox[8em]{\rule{0pt}{6ex}}\end{flushright} %49
\item 定価の3割引きの値段をxを用いて表せ。 \vfill \begin{flushright}\framebox[8em]{\rule{0pt}{6ex}}\end{flushright} %49
\item 「仕入れ値+利益=売った値段」の関係を使って方程式をたてて、xを求めよ。 \vfill \begin{flushright}\framebox[8em]{\rule{0pt}{6ex}}\end{flushright} %49
\end{enumerate}
\item 今年の生徒数は、去年に比べて4\%増加して156人になった。去年の生徒数を求めなさい。 \vfill \begin{flushright}\framebox[8em]{\rule{0pt}{6ex}}\end{flushright} %28
\item 80 円の鉛筆と 100 円のボールペンを合わせて 15 本買った。代金の合計は1340 円だった。それぞれ何本ずつ買ったのか。 \vfill \begin{flushright}\framebox[8em]{\rule{0pt}{6ex}}\end{flushright} %1
\end{enumerate}
    \vfill
\end{itemize}
\clearpage
 \begin{center}
   \LARGE\textbf{宮プリ ~一次方程式特訓編~\textcircled{\scriptsize 11}}
     \begin{flushright}
       名前\underline{\hspace{8zw}}
     \end{flushright}
 \end{center}

 \begin{itemize}
   \item 次の一次方程式をxについて解け
   \begin{enumerate}
\item $\frac{2}{3} x = -8$ \begin{flushright}\framebox[8em]{\rule{0pt}{6ex}}\end{flushright} %19
\item $1.9x+0.8=4x-2.7$ \begin{flushright}\framebox[8em]{\rule{0pt}{6ex}}\end{flushright} %97
\item $\frac{1}{2} x-6=-1$ \begin{flushright}\framebox[8em]{\rule{0pt}{6ex}}\end{flushright} %30
\item $0.2(x+6)=0.5x-0.6$ \begin{flushright}\framebox[8em]{\rule{0pt}{6ex}}\end{flushright} %104
\item $6+x=-5$ \begin{flushright}\framebox[8em]{\rule{0pt}{6ex}}\end{flushright} %11
\item $6(1-x)=-x+11$ \begin{flushright}\framebox[8em]{\rule{0pt}{6ex}}\end{flushright} %62
\item $\frac{2}{3} x+ \frac{3}{4} = \frac{1}{2} - \frac{1}{2}x$ \begin{flushright}\framebox[8em]{\rule{0pt}{6ex}}\end{flushright} %134
\item ある品物を仕入れて原価の40\%の利益を見込んで定価をつけたが売れなかったので、安売りの日に定価の20\%引きで売ったら480円の利益を得た。この品物の原価を求めなさい。 \vfill \begin{flushright}\framebox[8em]{\rule{0pt}{6ex}}\end{flushright} %29
\item 50の3割を求めよ \vfill \begin{flushright}\framebox[8em]{\rule{0pt}{6ex}}\end{flushright} %32
\item aの3\%を文字式で表せ。 \vfill \begin{flushright}\framebox[8em]{\rule{0pt}{6ex}}\end{flushright} %36
\end{enumerate}
    \vfill
\end{itemize}
\clearpage
 \begin{center}
   \LARGE\textbf{宮プリ ~一次方程式特訓編~\textcircled{\scriptsize 11}}
     \begin{flushright}
       名前\underline{\hspace{8zw}}
     \end{flushright}
 \end{center}

 \begin{itemize}
   \item 次の一次方程式をxについて解け
   \begin{enumerate}
\item $11- \frac{4}{5} x= \frac{2}{3} x$ \begin{flushright}\framebox[8em]{\rule{0pt}{6ex}}\end{flushright} %113
\item $2(4x+1)=3(3x-5)$ \begin{flushright}\framebox[8em]{\rule{0pt}{6ex}}\end{flushright} %70
\item $9x-8=5x+6$ \begin{flushright}\framebox[8em]{\rule{0pt}{6ex}}\end{flushright} %54
\item $5-x=4$ \begin{flushright}\framebox[8em]{\rule{0pt}{6ex}}\end{flushright} %16
\item $5x=2(x-9)$ \begin{flushright}\framebox[8em]{\rule{0pt}{6ex}}\end{flushright} %60
\item $-32x=-8$ \begin{flushright}\framebox[8em]{\rule{0pt}{6ex}}\end{flushright} %123
\item $16-9x=-23$ \begin{flushright}\framebox[8em]{\rule{0pt}{6ex}}\end{flushright} %33
\item ある数と 5 との和の 3 倍はもとの数の 7 倍から 1 を引いたものと等しい。もとの数を求めよ。 \vfill \begin{flushright}\framebox[8em]{\rule{0pt}{6ex}}\end{flushright} %3
\item 折り紙を子供に配る。3枚ずつ配ると20枚あまり、4枚ずつ配ると5枚足りなくなる。子供の人数と折り紙の枚数を求めなさい。 \vfill \begin{flushright}\framebox[8em]{\rule{0pt}{6ex}}\end{flushright} %7
\item 36人のクラスがある。女子の平均点が79点で、男子の平均点は70点でした。クラスの平均点が75点でした。男子の人数と女子の人数を求めなさい。 \vfill \begin{flushright}\framebox[8em]{\rule{0pt}{6ex}}\end{flushright} %30
\end{enumerate}
    \vfill
\end{itemize}
\clearpage
 \begin{center}
   \LARGE\textbf{宮プリ ~一次方程式特訓編~\textcircled{\scriptsize 11}}
     \begin{flushright}
       名前\underline{\hspace{8zw}}
     \end{flushright}
 \end{center}

 \begin{itemize}
   \item 次の一次方程式をxについて解け
   \begin{enumerate}
\item $8+5x=3(x-10)$ \begin{flushright}\framebox[8em]{\rule{0pt}{6ex}}\end{flushright} %64
\item $\frac{1}{6}x+3= \frac{3}{2} + \frac{3}{2} x$ \begin{flushright}\framebox[8em]{\rule{0pt}{6ex}}\end{flushright} %119
\item $\frac{5x+4}{6} = \frac{3}{4}x+1$ \begin{flushright}\framebox[8em]{\rule{0pt}{6ex}}\end{flushright} %146
\item $3.3x+1.4=0.6x-1.3$ \begin{flushright}\framebox[8em]{\rule{0pt}{6ex}}\end{flushright} %98
\item $2x+5=8x+4$ \begin{flushright}\framebox[8em]{\rule{0pt}{6ex}}\end{flushright} %52
\item $- \frac{7}{9} x= \frac{28}{3}$ \begin{flushright}\framebox[8em]{\rule{0pt}{6ex}}\end{flushright} %107
\item $2(x+4)=5(x-1)$ \begin{flushright}\framebox[8em]{\rule{0pt}{6ex}}\end{flushright} %80
\item 650のx割を文字式で表せ \vfill \begin{flushright}\framebox[8em]{\rule{0pt}{6ex}}\end{flushright} %35
\item 時速3kmで2時間歩いたときの道のりは何kmか。 \vfill \begin{flushright}\framebox[8em]{\rule{0pt}{6ex}}\end{flushright} %11
\item 一昨年に比べて昨年は生徒数が10人増加した。昨年にくらべて今年は生徒数が5\%減少した。今年の生徒数は228人だった。 一昨年の生徒数をx人として問いに答えよ。 \vfill \begin{flushright}\framebox[8em]{\rule{0pt}{6ex}}\end{flushright} %48
\begin{enumerate}
\item 「一昨年に比べて昨年は生徒数が10人増加した。」ことから  \vfill \begin{flushright}\framebox[8em]{\rule{0pt}{6ex}}\end{flushright} %48
\item 昨年の生徒数をxを用いて表せ。 \vfill \begin{flushright}\framebox[8em]{\rule{0pt}{6ex}}\end{flushright} %48
\item 「昨年にくらべて今年は生徒数が5\%減少した。」ことから今年の生徒数をxを用いて表せ。 \vfill \begin{flushright}\framebox[8em]{\rule{0pt}{6ex}}\end{flushright} %48
\item 方程式をたててxを求めよ。 \vfill \begin{flushright}\framebox[8em]{\rule{0pt}{6ex}}\end{flushright} %48
\end{enumerate}
\end{enumerate}
    \vfill
\end{itemize}
\clearpage
 \begin{center}
   \LARGE\textbf{宮プリ ~一次方程式特訓編~\textcircled{\scriptsize 11}}
     \begin{flushright}
       名前\underline{\hspace{8zw}}
     \end{flushright}
 \end{center}

 \begin{itemize}
   \item 次の一次方程式をxについて解け
   \begin{enumerate}
\item $2(x+3)=x+11$ \begin{flushright}\framebox[8em]{\rule{0pt}{6ex}}\end{flushright} %78
\item $\frac{3x+5}{4} -2x= {2x-7}{6} - \frac{3}{4}$ \begin{flushright}\framebox[8em]{\rule{0pt}{6ex}}\end{flushright} %152
\item $8=-2(3x+2)$ \begin{flushright}\framebox[8em]{\rule{0pt}{6ex}}\end{flushright} %61
\item $3x-7=-16$ \begin{flushright}\framebox[8em]{\rule{0pt}{6ex}}\end{flushright} %15
\item $14+9x=-4$ \begin{flushright}\framebox[8em]{\rule{0pt}{6ex}}\end{flushright} %26
\item $\frac{1}{2}x- \frac{1}{4} = \frac{3}{4} x$ \begin{flushright}\framebox[8em]{\rule{0pt}{6ex}}\end{flushright} %114
\item $52x+1=34x$ \begin{flushright}\framebox[8em]{\rule{0pt}{6ex}}\end{flushright} %126
\item 弟が家を出て毎分40mで歩く。その5分後に兄が毎分60mで追いかける。兄が弟に追いつくのは家から何mの地点か。 \vfill \begin{flushright}\framebox[8em]{\rule{0pt}{6ex}}\end{flushright} %14
\item ビンの中にいくらかの水があった。はじめにそのうちの200mlを飲んだ。さらに残っている水の30\%を飲んだ。 するとビンの中には210mlの水が残った。はじめにビンの中にはあった水をxmlとして問いに答えよ。 \vfill \begin{flushright}\framebox[8em]{\rule{0pt}{6ex}}\end{flushright} %40
\begin{enumerate}
\item 1回めに水を飲んだ後に残っている水の量をxを使って表わせ。 \vfill \begin{flushright}\framebox[8em]{\rule{0pt}{6ex}}\end{flushright} %40
\item 2回めに水を飲んだ後に残っている水の量をxを使って表わせ。 \vfill \begin{flushright}\framebox[8em]{\rule{0pt}{6ex}}\end{flushright} %40
\item 方程式をたててxを求めよ。 \vfill \begin{flushright}\framebox[8em]{\rule{0pt}{6ex}}\end{flushright} %40
\end{enumerate}
\item aの9割を文字式で表せ \vfill \begin{flushright}\framebox[8em]{\rule{0pt}{6ex}}\end{flushright} %37
\end{enumerate}
    \vfill
\end{itemize}
\clearpage
 \begin{center}
   \LARGE\textbf{宮プリ ~一次方程式特訓編~\textcircled{\scriptsize 11}}
     \begin{flushright}
       名前\underline{\hspace{8zw}}
     \end{flushright}
 \end{center}

 \begin{itemize}
   \item 次の一次方程式をxについて解け
   \begin{enumerate}
\item $6x-8=4x$ \begin{flushright}\framebox[8em]{\rule{0pt}{6ex}}\end{flushright} %36
\item $\frac{2}{3} x+ \frac{1}{4} = \frac{1}{2}x-1$ \begin{flushright}\framebox[8em]{\rule{0pt}{6ex}}\end{flushright} %161
\item $0.3x+1.7=0.2$ \begin{flushright}\framebox[8em]{\rule{0pt}{6ex}}\end{flushright} %96
\item $1.12-0.67x=0.06-1.2x$ \begin{flushright}\framebox[8em]{\rule{0pt}{6ex}}\end{flushright} %101
\item $\frac{1}{2} x-  = \frac{5}{4} -x$ \begin{flushright}\framebox[8em]{\rule{0pt}{6ex}}\end{flushright} %130
\item $\frac{5}{6} x=-25$ \begin{flushright}\framebox[8em]{\rule{0pt}{6ex}}\end{flushright} %106
\item $17-6x=8x-4$ \begin{flushright}\framebox[8em]{\rule{0pt}{6ex}}\end{flushright} %57
\item 時速30kmで2時間進むと、距離は何kmになるか。 \vfill \begin{flushright}\framebox[8em]{\rule{0pt}{6ex}}\end{flushright} %9
\item 300mを5分で歩いた。分速何mか。 \vfill \begin{flushright}\framebox[8em]{\rule{0pt}{6ex}}\end{flushright} %8
\item 2400mの道のりを分速80mの一定の速さで歩いたら、かかる時間は何分か。 \vfill \begin{flushright}\framebox[8em]{\rule{0pt}{6ex}}\end{flushright} %13
\end{enumerate}
    \vfill
\end{itemize}
\clearpage
 \begin{center}
   \LARGE\textbf{宮プリ ~一次方程式特訓編~\textcircled{\scriptsize 11}}
     \begin{flushright}
       名前\underline{\hspace{8zw}}
     \end{flushright}
 \end{center}

 \begin{itemize}
   \item 次の一次方程式をxについて解け
   \begin{enumerate}
\item $\frac{x+5}{3} =x-3$ \begin{flushright}\framebox[8em]{\rule{0pt}{6ex}}\end{flushright} %142
\item $0.1x+0.3=1.8$ \begin{flushright}\framebox[8em]{\rule{0pt}{6ex}}\end{flushright} %157
\item $4x+1=3(2x-9)+4$ \begin{flushright}\framebox[8em]{\rule{0pt}{6ex}}\end{flushright} %68
\item $23x=-12x-1$ \begin{flushright}\framebox[8em]{\rule{0pt}{6ex}}\end{flushright} %127
\item $x+2=2$ \begin{flushright}\framebox[8em]{\rule{0pt}{6ex}}\end{flushright} %8
\item $3(x-2)=x+12$ \begin{flushright}\framebox[8em]{\rule{0pt}{6ex}}\end{flushright} %79
\item $\frac{6}{5} x- \frac{9}{10} = \frac{7}{5} x-3$ \begin{flushright}\framebox[8em]{\rule{0pt}{6ex}}\end{flushright} %133
\item 花子さんが家をでて毎分40mで歩いていった。その10分後に母が毎分120mで花子さんを追いかけた。母が花子さんに追いつくのは花子さんが家を出てから何分後か。 \vfill \begin{flushright}\framebox[8em]{\rule{0pt}{6ex}}\end{flushright} %15
\item ある商品に原価の3割の利益を見込んで定価をつけたら、定価が910円になった。この商品の原価を求めよ。 \vfill \begin{flushright}\framebox[8em]{\rule{0pt}{6ex}}\end{flushright} %44
\item ある中学校の全生徒数は796人です。女子の人数は男子の人数の99\%です。この中学校の男子の人数と女子の人数をそれぞれ求めなさい。 \vfill \begin{flushright}\framebox[8em]{\rule{0pt}{6ex}}\end{flushright} %42
\end{enumerate}
    \vfill
\end{itemize}
\clearpage
 \begin{center}
   \LARGE\textbf{宮プリ ~一次方程式特訓編~\textcircled{\scriptsize 11}}
     \begin{flushright}
       名前\underline{\hspace{8zw}}
     \end{flushright}
 \end{center}

 \begin{itemize}
   \item 次の一次方程式をxについて解け
   \begin{enumerate}
\item $11-4x=2x-9$ \begin{flushright}\framebox[8em]{\rule{0pt}{6ex}}\end{flushright} %50
\item $34x=12$ \begin{flushright}\framebox[8em]{\rule{0pt}{6ex}}\end{flushright} %124
\item $1.7-0.31x=0.19x+0.2$ \begin{flushright}\framebox[8em]{\rule{0pt}{6ex}}\end{flushright} %95
\item $\frac{x}{8} -2= \frac{9-x}{6}$ \begin{flushright}\framebox[8em]{\rule{0pt}{6ex}}\end{flushright} %147
\item $0.5(x-9)=-1.3x$ \begin{flushright}\framebox[8em]{\rule{0pt}{6ex}}\end{flushright} %102
\item $\frac{1}{4} x+ \frac{3}{2} = \frac{1}{3} x$ \begin{flushright}\framebox[8em]{\rule{0pt}{6ex}}\end{flushright} %111
\item $8(2x+1)=4(x-3)$ \begin{flushright}\framebox[8em]{\rule{0pt}{6ex}}\end{flushright} %81
\item 全校生徒300人のうち、自転車通学の割合は男子が全男子数の2割で、女子は全女子数の1割である。 自転車通学の人数は男女合わせて46である。全男子数をx人として次の問いに答えよ。 \vfill \begin{flushright}\framebox[8em]{\rule{0pt}{6ex}}\end{flushright} %38
\begin{enumerate}
\item 全女子数をxを使って表わせ。 \vfill \begin{flushright}\framebox[8em]{\rule{0pt}{6ex}}\end{flushright} %38
\item 自転車通学している男子の人数をxを使って表わせ。 \vfill \begin{flushright}\framebox[8em]{\rule{0pt}{6ex}}\end{flushright} %38
\item 自転車通学している女子の人数をxを使って表わせ。 \vfill \begin{flushright}\framebox[8em]{\rule{0pt}{6ex}}\end{flushright} %38
\item 方程式をつくってxを求めよ。 \vfill \begin{flushright}\framebox[8em]{\rule{0pt}{6ex}}\end{flushright} %38
\end{enumerate}
\item 濃度のわからない食塩水が200gある。ここに10\%の食塩水を300g混ぜたら8\%の食塩水ができた。はじめにあった食塩水の濃度は何\%だったか。 \vfill \begin{flushright}\framebox[8em]{\rule{0pt}{6ex}}\end{flushright} %21
\item はじめ姉がいくつかのアメを持っており、妹は持っていなかった。姉は自分の持っていたアメの6割を妹にあげた。 姉は残ったアメのうち6個を食べた。妹はもらったアメの5割を食べた。すると姉と妹の持っているアメの数が同数になった。 姉がはじめに持っていたアメの数をxとして次の問いに答えよ。 \vfill \begin{flushright}\framebox[8em]{\rule{0pt}{6ex}}\end{flushright} %39
\begin{enumerate}
\item 姉が妹にあげたアメの数をxを使って表わせ。 \vfill \begin{flushright}\framebox[8em]{\rule{0pt}{6ex}}\end{flushright} %39
\item 姉が妹にアメをあげたあと、残ったアメの数をxを使って表わせ。 \vfill \begin{flushright}\framebox[8em]{\rule{0pt}{6ex}}\end{flushright} %39
\item 妹がアメを食べた後、残った妹のアメの数をxを使って表わせ。 \vfill \begin{flushright}\framebox[8em]{\rule{0pt}{6ex}}\end{flushright} %39
\item 姉がアメを食べた後、残った姉のアメの数をxを使って表わせ。 \vfill \begin{flushright}\framebox[8em]{\rule{0pt}{6ex}}\end{flushright} %39
\item 方程式をつくってxを求めよ。 \vfill \begin{flushright}\framebox[8em]{\rule{0pt}{6ex}}\end{flushright} %39
\end{enumerate}
\end{enumerate}
    \vfill
\end{itemize}
\clearpage
\end{document}
\end{document}