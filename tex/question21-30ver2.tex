\documentclass[a4paper,fleqn,papersize,15pt]{jsarticle}

\setlength{\topmargin}{-1in}
\addtolength{\topmargin}{5mm}
\setlength{\headheight}{5mm}
\setlength{\headsep}{0mm}
\usepackage{color}

\setlength{\textheight}{\paperheight}
\addtolength{\textheight}{-25mm}
\setlength{\footskip}{5mm}
\renewcommand{\labelenumi}{(\theenumi)}
\begin{document}
 \begin{center}
   \LARGE\textbf{宮プリ ~一次方程式特訓編~\textcircled{\scriptsize 11}}
     \begin{flushright}
       名前\underline{\hspace{8zw}}
     \end{flushright}
 \end{center}

 \begin{itemize}
   \item 次の一次方程式をxについて解け
   \begin{enumerate}
\item $\frac{1}{2}x-3=12$ \begin{flushright}\textcolor{red}{\framebox[8em]{\rule{0pt}{6ex}}\end{flushright}} \vfill %155
\item $- \frac{1}{3}(9x+24)+6(\frac{1}{3} x-1)= \frac{1}{4}(16x+8)+3x-1$ \begin{flushright}\textcolor{red}{\framebox[8em]{\rule{0pt}{6ex}}\end{flushright}} \vfill %89
\item $8=-2(3x+2)$ \begin{flushright}\textcolor{red}{\framebox[8em]{\rule{0pt}{6ex}}\end{flushright}} \vfill %61
\item $2x=1$ \begin{flushright}\textcolor{red}{\framebox[8em]{\rule{0pt}{6ex}}\end{flushright}} \vfill %18
\item $\frac{1}{6} - \frac{9}{8} x= \frac{3}{4} x- \frac{1}{2}$ \begin{flushright}\textcolor{red}{\framebox[8em]{\rule{0pt}{6ex}}\end{flushright}} \vfill %136
\item $4x=-16$ \begin{flushright}\textcolor{red}{\framebox[8em]{\rule{0pt}{6ex}}\end{flushright}} \vfill %13
\item $7x-3=9+4x$ \begin{flushright}\textcolor{red}{\framebox[8em]{\rule{0pt}{6ex}}\end{flushright}} \vfill %44
\item $52x+1=34x$ \begin{flushright}\textcolor{red}{\framebox[8em]{\rule{0pt}{6ex}}\end{flushright}} \vfill %126
\item $\frac{x}{6} = \frac{3-x}{4}$ \begin{flushright}\textcolor{red}{\framebox[8em]{\rule{0pt}{6ex}}\end{flushright}} \vfill %140
\item $-3x=-27$ \begin{flushright}\textcolor{red}{\framebox[8em]{\rule{0pt}{6ex}}\end{flushright}} \vfill %7
\end{enumerate}
    \vfill
\end{itemize}
\clearpage
 \begin{center}
   \LARGE\textbf{宮プリ ~一次方程式特訓編~\textcircled{\scriptsize 11}}
     \begin{flushright}
       名前\underline{\hspace{8zw}}
     \end{flushright}
 \end{center}

 \begin{itemize}
   \item 次の一次方程式をxについて解け
   \begin{enumerate}
\item $2+ \frac{3}{4} x= \frac{5}{6} x$ \begin{flushright}\textcolor{red}{\framebox[8em]{\rule{0pt}{6ex}}\end{flushright}} \vfill %112
\item $\frac{2}{3} x+ \frac{3}{4} = \frac{1}{2} - \frac{1}{2}x$ \begin{flushright}\textcolor{red}{\framebox[8em]{\rule{0pt}{6ex}}\end{flushright}} \vfill %134
\item $\frac{1}{2} x-1= \frac{1}{3}x+ \frac{5}{2}$ \begin{flushright}\textcolor{red}{\framebox[8em]{\rule{0pt}{6ex}}\end{flushright}} \vfill %159
\item $2x+15=-x$ \begin{flushright}\textcolor{red}{\framebox[8em]{\rule{0pt}{6ex}}\end{flushright}} \vfill %37
\item $14-15x=8-13x$ \begin{flushright}\textcolor{red}{\framebox[8em]{\rule{0pt}{6ex}}\end{flushright}} \vfill %55
\item $3(x-1)-2(x+3)=6-(2x+3)$ \begin{flushright}\textcolor{red}{\framebox[8em]{\rule{0pt}{6ex}}\end{flushright}} \vfill %163
\item $\frac{1}{2}x- \frac{1}{4} = \frac{3}{4} x$ \begin{flushright}\textcolor{red}{\framebox[8em]{\rule{0pt}{6ex}}\end{flushright}} \vfill %114
\item $3x=15$ \begin{flushright}\textcolor{red}{\framebox[8em]{\rule{0pt}{6ex}}\end{flushright}} \vfill %4
\item $3x=18-5x$ \begin{flushright}\textcolor{red}{\framebox[8em]{\rule{0pt}{6ex}}\end{flushright}} \vfill %39
\item $14+9x=-4$ \begin{flushright}\textcolor{red}{\framebox[8em]{\rule{0pt}{6ex}}\end{flushright}} \vfill %26
\end{enumerate}
    \vfill
\end{itemize}
\clearpage
 \begin{center}
   \LARGE\textbf{宮プリ ~一次方程式特訓編~\textcircled{\scriptsize 11}}
     \begin{flushright}
       名前\underline{\hspace{8zw}}
     \end{flushright}
 \end{center}

 \begin{itemize}
   \item 次の一次方程式をxについて解け
   \begin{enumerate}
\item $3x-7=-2x+9$ \begin{flushright}\textcolor{red}{\framebox[8em]{\rule{0pt}{6ex}}\end{flushright}} \vfill %53
\item $x+3=-8$ \begin{flushright}\textcolor{red}{\framebox[8em]{\rule{0pt}{6ex}}\end{flushright}} \vfill %2
\item $\frac{7x+8}{12} = \frac{5x-9}{8}$ \begin{flushright}\textcolor{red}{\framebox[8em]{\rule{0pt}{6ex}}\end{flushright}} \vfill %148
\item $9-2(x-4)=3x+7$ \begin{flushright}\textcolor{red}{\framebox[8em]{\rule{0pt}{6ex}}\end{flushright}} \vfill %77
\item $\frac{4}{3} x+7=-2$ \begin{flushright}\textcolor{red}{\framebox[8em]{\rule{0pt}{6ex}}\end{flushright}} \vfill %31
\item $4x-7=8$ \begin{flushright}\textcolor{red}{\framebox[8em]{\rule{0pt}{6ex}}\end{flushright}} \vfill %29
\item $1.7x-0.7=1.6x-0.5$ \begin{flushright}\textcolor{red}{\framebox[8em]{\rule{0pt}{6ex}}\end{flushright}} \vfill %94
\item $x-4=7$ \begin{flushright}\textcolor{red}{\framebox[8em]{\rule{0pt}{6ex}}\end{flushright}} \vfill %1
\item $0.4x+0.6=0.2$ \begin{flushright}\textcolor{red}{\framebox[8em]{\rule{0pt}{6ex}}\end{flushright}} \vfill %90
\item $\frac{6}{5} x+ \frac{5}{2} =- \frac{3}{10} x-2$ \begin{flushright}\textcolor{red}{\framebox[8em]{\rule{0pt}{6ex}}\end{flushright}} \vfill %116
\end{enumerate}
    \vfill
\end{itemize}
\clearpage
 \begin{center}
   \LARGE\textbf{宮プリ ~一次方程式特訓編~\textcircled{\scriptsize 11}}
     \begin{flushright}
       名前\underline{\hspace{8zw}}
     \end{flushright}
 \end{center}

 \begin{itemize}
   \item 次の一次方程式をxについて解け
   \begin{enumerate}
\item $11- \frac{4}{5} x= \frac{2}{3} x$ \begin{flushright}\textcolor{red}{\framebox[8em]{\rule{0pt}{6ex}}\end{flushright}} \vfill %113
\item $1.12-0.67x=0.06-1.2x$ \begin{flushright}\textcolor{red}{\framebox[8em]{\rule{0pt}{6ex}}\end{flushright}} \vfill %101
\item $4x+7=x-11$ \begin{flushright}\textcolor{red}{\framebox[8em]{\rule{0pt}{6ex}}\end{flushright}} \vfill %42
\item $\frac{2}{3} x+ \frac{2}{9} =2- \frac{2}{9} x$ \begin{flushright}\textcolor{red}{\framebox[8em]{\rule{0pt}{6ex}}\end{flushright}} \vfill %118
\item $2(3x-5)=3(x+2)+4(2x-1)$ \begin{flushright}\textcolor{red}{\framebox[8em]{\rule{0pt}{6ex}}\end{flushright}} \vfill %86
\item $2+3x=x+14$ \begin{flushright}\textcolor{red}{\framebox[8em]{\rule{0pt}{6ex}}\end{flushright}} \vfill %49
\item $3.3x+1.4=0.6x-1.3$ \begin{flushright}\textcolor{red}{\framebox[8em]{\rule{0pt}{6ex}}\end{flushright}} \vfill %98
\item $0.6x+0.4(3x-2)=4.6$ \begin{flushright}\textcolor{red}{\framebox[8em]{\rule{0pt}{6ex}}\end{flushright}} \vfill %105
\item $0.1x+0.3=1.8$ \begin{flushright}\textcolor{red}{\framebox[8em]{\rule{0pt}{6ex}}\end{flushright}} \vfill %157
\item $x-8=-3$ \begin{flushright}\textcolor{red}{\framebox[8em]{\rule{0pt}{6ex}}\end{flushright}} \vfill %3
\end{enumerate}
    \vfill
\end{itemize}
\clearpage
 \begin{center}
   \LARGE\textbf{宮プリ ~一次方程式特訓編~\textcircled{\scriptsize 11}}
     \begin{flushright}
       名前\underline{\hspace{8zw}}
     \end{flushright}
 \end{center}

 \begin{itemize}
   \item 次の一次方程式をxについて解け
   \begin{enumerate}
\item $-3(4-x)=x$ \begin{flushright}\textcolor{red}{\framebox[8em]{\rule{0pt}{6ex}}\end{flushright}} \vfill %59
\item $5-x=4$ \begin{flushright}\textcolor{red}{\framebox[8em]{\rule{0pt}{6ex}}\end{flushright}} \vfill %16
\item $\frac{5x+4}{6} = \frac{3}{4}x+1$ \begin{flushright}\textcolor{red}{\framebox[8em]{\rule{0pt}{6ex}}\end{flushright}} \vfill %146
\item $1.2x+1.2=0.7x-1.3$ \begin{flushright}\textcolor{red}{\framebox[8em]{\rule{0pt}{6ex}}\end{flushright}} \vfill %164
\item $2x=x-11$ \begin{flushright}\textcolor{red}{\framebox[8em]{\rule{0pt}{6ex}}\end{flushright}} \vfill %34
\item $\frac{1}{2}(4x-8)=5x+2$ \begin{flushright}\textcolor{red}{\framebox[8em]{\rule{0pt}{6ex}}\end{flushright}} \vfill %84
\item $\frac{1}{2} x- \frac{4}{5} = \frac{4}{7} x+ \frac{3}{10}$ \begin{flushright}\textcolor{red}{\framebox[8em]{\rule{0pt}{6ex}}\end{flushright}} \vfill %135
\item $5-x=3x+13$ \begin{flushright}\textcolor{red}{\framebox[8em]{\rule{0pt}{6ex}}\end{flushright}} \vfill %43
\item $3x+2(5-x)=6$ \begin{flushright}\textcolor{red}{\framebox[8em]{\rule{0pt}{6ex}}\end{flushright}} \vfill %66
\item $8x=2x+9$ \begin{flushright}\textcolor{red}{\framebox[8em]{\rule{0pt}{6ex}}\end{flushright}} \vfill %38
\end{enumerate}
    \vfill
\end{itemize}
\clearpage
 \begin{center}
   \LARGE\textbf{宮プリ ~一次方程式特訓編~\textcircled{\scriptsize 11}}
     \begin{flushright}
       名前\underline{\hspace{8zw}}
     \end{flushright}
 \end{center}

 \begin{itemize}
   \item 次の一次方程式をxについて解け
   \begin{enumerate}
\item $\frac{3}{4} x-1=2$ \begin{flushright}\textcolor{red}{\framebox[8em]{\rule{0pt}{6ex}}\end{flushright}} \vfill %154
\item $7x+5=11$ \begin{flushright}\textcolor{red}{\framebox[8em]{\rule{0pt}{6ex}}\end{flushright}} \vfill %46
\item $7x-8=4(4+x)$ \begin{flushright}\textcolor{red}{\framebox[8em]{\rule{0pt}{6ex}}\end{flushright}} \vfill %65
\item $\frac{x}{8} -2= \frac{9-x}{6}$ \begin{flushright}\textcolor{red}{\framebox[8em]{\rule{0pt}{6ex}}\end{flushright}} \vfill %147
\item $- \frac{7}{9} x= \frac{28}{3}$ \begin{flushright}\textcolor{red}{\framebox[8em]{\rule{0pt}{6ex}}\end{flushright}} \vfill %107
\item $1.9x+0.8=4x-2.7$ \begin{flushright}\textcolor{red}{\framebox[8em]{\rule{0pt}{6ex}}\end{flushright}} \vfill %97
\item $- \frac{3}{4}x = 9$ \begin{flushright}\textcolor{red}{\framebox[8em]{\rule{0pt}{6ex}}\end{flushright}} \vfill %20
\item $4x+1=-11$ \begin{flushright}\textcolor{red}{\framebox[8em]{\rule{0pt}{6ex}}\end{flushright}} \vfill %22
\item $0.38-0.18x=0.2$ \begin{flushright}\textcolor{red}{\framebox[8em]{\rule{0pt}{6ex}}\end{flushright}} \vfill %93
\item $\frac{1}{2} x-  = \frac{5}{4} -x$ \begin{flushright}\textcolor{red}{\framebox[8em]{\rule{0pt}{6ex}}\end{flushright}} \vfill %130
\end{enumerate}
    \vfill
\end{itemize}
\clearpage
 \begin{center}
   \LARGE\textbf{宮プリ ~一次方程式特訓編~\textcircled{\scriptsize 11}}
     \begin{flushright}
       名前\underline{\hspace{8zw}}
     \end{flushright}
 \end{center}

 \begin{itemize}
   \item 次の一次方程式をxについて解け
   \begin{enumerate}
\item $\frac{2}{3} (9-6x)-4(x+ \frac{1}{2} )=- \frac{3}{2} (8x-12)$ \begin{flushright}\textcolor{red}{\framebox[8em]{\rule{0pt}{6ex}}\end{flushright}} \vfill %88
\item $ 4-x= \frac{2x+7}{5}$ \begin{flushright}\textcolor{red}{\framebox[8em]{\rule{0pt}{6ex}}\end{flushright}} \vfill %143
\item $0.5(x-9)=-1.3x$ \begin{flushright}\textcolor{red}{\framebox[8em]{\rule{0pt}{6ex}}\end{flushright}} \vfill %102
\item $23x=-12x-1$ \begin{flushright}\textcolor{red}{\framebox[8em]{\rule{0pt}{6ex}}\end{flushright}} \vfill %127
\item $ 2x-1= \frac{5x+11}{8}$ \begin{flushright}\textcolor{red}{\framebox[8em]{\rule{0pt}{6ex}}\end{flushright}} \vfill %145
\item $5+3x=4$ \begin{flushright}\textcolor{red}{\framebox[8em]{\rule{0pt}{6ex}}\end{flushright}} \vfill %47
\item $-4x+5=-11$ \begin{flushright}\textcolor{red}{\framebox[8em]{\rule{0pt}{6ex}}\end{flushright}} \vfill %25
\item $6(1-x)=-x+11$ \begin{flushright}\textcolor{red}{\framebox[8em]{\rule{0pt}{6ex}}\end{flushright}} \vfill %62
\item $1.1x+0.6=0.85x-1.15$ \begin{flushright}\textcolor{red}{\framebox[8em]{\rule{0pt}{6ex}}\end{flushright}} \vfill %99
\item $13+4x=25$ \begin{flushright}\textcolor{red}{\framebox[8em]{\rule{0pt}{6ex}}\end{flushright}} \vfill %27
\end{enumerate}
    \vfill
\end{itemize}
\clearpage
 \begin{center}
   \LARGE\textbf{宮プリ ~一次方程式特訓編~\textcircled{\scriptsize 11}}
     \begin{flushright}
       名前\underline{\hspace{8zw}}
     \end{flushright}
 \end{center}

 \begin{itemize}
   \item 次の一次方程式をxについて解け
   \begin{enumerate}
\item $8-5x=-12$ \begin{flushright}\textcolor{red}{\framebox[8em]{\rule{0pt}{6ex}}\end{flushright}} \vfill %24
\item $2(x+4)=5(x-1)$ \begin{flushright}\textcolor{red}{\framebox[8em]{\rule{0pt}{6ex}}\end{flushright}} \vfill %80
\item $2x+1=x+7$ \begin{flushright}\textcolor{red}{\framebox[8em]{\rule{0pt}{6ex}}\end{flushright}} \vfill %40
\item $34x+12=74$ \begin{flushright}\textcolor{red}{\framebox[8em]{\rule{0pt}{6ex}}\end{flushright}} \vfill %128
\item $6x-8=4x$ \begin{flushright}\textcolor{red}{\framebox[8em]{\rule{0pt}{6ex}}\end{flushright}} \vfill %36
\item $6+x=-5$ \begin{flushright}\textcolor{red}{\framebox[8em]{\rule{0pt}{6ex}}\end{flushright}} \vfill %11
\item $16-3x=4x-2(5x+1)$ \begin{flushright}\textcolor{red}{\framebox[8em]{\rule{0pt}{6ex}}\end{flushright}} \vfill %69
\item $0.01x+0.98=0.5x$ \begin{flushright}\textcolor{red}{\framebox[8em]{\rule{0pt}{6ex}}\end{flushright}} \vfill %92
\item $\frac{3}{4} x-1=5$ \begin{flushright}\textcolor{red}{\framebox[8em]{\rule{0pt}{6ex}}\end{flushright}} \vfill %108
\item $4x+1=3(2x-9)+4$ \begin{flushright}\textcolor{red}{\framebox[8em]{\rule{0pt}{6ex}}\end{flushright}} \vfill %68
\end{enumerate}
    \vfill
\end{itemize}
\clearpage
 \begin{center}
   \LARGE\textbf{宮プリ ~一次方程式特訓編~\textcircled{\scriptsize 11}}
     \begin{flushright}
       名前\underline{\hspace{8zw}}
     \end{flushright}
 \end{center}

 \begin{itemize}
   \item 次の一次方程式をxについて解け
   \begin{enumerate}
\item $5(x+1)=3x-8$ \begin{flushright}\textcolor{red}{\framebox[8em]{\rule{0pt}{6ex}}\end{flushright}} \vfill %82
\item $\frac{x-1}{9} = \frac{1-2x}{6}$ \begin{flushright}\textcolor{red}{\framebox[8em]{\rule{0pt}{6ex}}\end{flushright}} \vfill %149
\item $1.7-0.31x=0.19x+0.2$ \begin{flushright}\textcolor{red}{\framebox[8em]{\rule{0pt}{6ex}}\end{flushright}} \vfill %95
\item $34x=12$ \begin{flushright}\textcolor{red}{\framebox[8em]{\rule{0pt}{6ex}}\end{flushright}} \vfill %124
\item $2(3x-2)=5x-4$ \begin{flushright}\textcolor{red}{\framebox[8em]{\rule{0pt}{6ex}}\end{flushright}} \vfill %63
\item $3x-2=-17$ \begin{flushright}\textcolor{red}{\framebox[8em]{\rule{0pt}{6ex}}\end{flushright}} \vfill %23
\item $56-23x=43$ \begin{flushright}\textcolor{red}{\framebox[8em]{\rule{0pt}{6ex}}\end{flushright}} \vfill %129
\item $ 2(4x+ \frac{1}{2})= \frac{3}{4}(8x-12)$ \begin{flushright}\textcolor{red}{\framebox[8em]{\rule{0pt}{6ex}}\end{flushright}} \vfill %85
\item $0.1x-0.3=0.2x+1.2$ \begin{flushright}\textcolor{red}{\framebox[8em]{\rule{0pt}{6ex}}\end{flushright}} \vfill %158
\item $3(x-2)=x+12$ \begin{flushright}\textcolor{red}{\framebox[8em]{\rule{0pt}{6ex}}\end{flushright}} \vfill %79
\end{enumerate}
    \vfill
\end{itemize}
\clearpage
 \begin{center}
   \LARGE\textbf{宮プリ ~一次方程式特訓編~\textcircled{\scriptsize 11}}
     \begin{flushright}
       名前\underline{\hspace{8zw}}
     \end{flushright}
 \end{center}

 \begin{itemize}
   \item 次の一次方程式をxについて解け
   \begin{enumerate}
\item $0.1x=0.5x-1.6$ \begin{flushright}\textcolor{red}{\framebox[8em]{\rule{0pt}{6ex}}\end{flushright}} \vfill %91
\item $\frac{3}{4} x= \frac{2x-5}{3}$ \begin{flushright}\textcolor{red}{\framebox[8em]{\rule{0pt}{6ex}}\end{flushright}} \vfill %141
\item $-32x=-8$ \begin{flushright}\textcolor{red}{\framebox[8em]{\rule{0pt}{6ex}}\end{flushright}} \vfill %123
\item $\frac{2}{3} x+ \frac{2}{9} = \frac{1}{2}x+1$ \begin{flushright}\textcolor{red}{\framebox[8em]{\rule{0pt}{6ex}}\end{flushright}} \vfill %165
\item $\frac{x+5}{3} =x-3$ \begin{flushright}\textcolor{red}{\framebox[8em]{\rule{0pt}{6ex}}\end{flushright}} \vfill %142
\item $- \frac{7x+8}{9} = \frac{10-x}{6} -2x+ \frac{4}{3}$ \begin{flushright}\textcolor{red}{\framebox[8em]{\rule{0pt}{6ex}}\end{flushright}} \vfill %153
\item $2(x+1)=3x+4$ \begin{flushright}\textcolor{red}{\framebox[8em]{\rule{0pt}{6ex}}\end{flushright}} \vfill %74
\item $11-4x=2x-9$ \begin{flushright}\textcolor{red}{\framebox[8em]{\rule{0pt}{6ex}}\end{flushright}} \vfill %50
\item $x+2=2$ \begin{flushright}\textcolor{red}{\framebox[8em]{\rule{0pt}{6ex}}\end{flushright}} \vfill %8
\item $\frac{1}{3} x- \frac{5}{8} = \frac{3}{4} x- \frac{3}{8}$ \begin{flushright}\textcolor{red}{\framebox[8em]{\rule{0pt}{6ex}}\end{flushright}} \vfill %121
\end{enumerate}
    \vfill
\end{itemize}
\clearpage
\end{document}
\end{document}
