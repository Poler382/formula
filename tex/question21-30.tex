\documentclass[a4paper,fleqn,papersize,15pt]{jsarticle}

\setlength{\topmargin}{-1in}
\addtolength{\topmargin}{5mm}
\setlength{\headheight}{5mm}
\setlength{\headsep}{0mm}
\usepackage{color}

\setlength{\textheight}{\paperheight}
\addtolength{\textheight}{-25mm}
\setlength{\footskip}{5mm}
\renewcommand{\labelenumi}{(\theenumi)}
\begin{document}
 \begin{center}
   \LARGE\textbf{宮プリ ~一次方程式特訓編~\textcircled{\scriptsize 11}}
     \begin{flushright}
       名前\underline{\hspace{8zw}}
     \end{flushright}
 \end{center}

 \begin{itemize}
   \item 次の一次方程式をxについて解け
   \begin{enumerate}
\item $\frac{1}{2}x-3=12$ \begin{flushright}\textcolor{red}{\framebox[8em]{\rule{0pt}{6ex}}\end{flushright}} %155
\item $- \frac{1}{3}(9x+24)+6(\frac{1}{3} x-1)= \frac{1}{4}(16x+8)+3x-1$ \begin{flushright}\textcolor{red}{\framebox[8em]{\rule{0pt}{6ex}}\end{flushright}} %89
\item $8=-2(3x+2)$ \begin{flushright}\textcolor{red}{\framebox[8em]{\rule{0pt}{6ex}}\end{flushright}} %61
\item $2x=1$ \begin{flushright}\textcolor{red}{\framebox[8em]{\rule{0pt}{6ex}}\end{flushright}} %18
\item $\frac{1}{6} - \frac{9}{8} x= \frac{3}{4} x- \frac{1}{2}$ \begin{flushright}\textcolor{red}{\framebox[8em]{\rule{0pt}{6ex}}\end{flushright}} %136
\item $4x=-16$ \begin{flushright}\textcolor{red}{\framebox[8em]{\rule{0pt}{6ex}}\end{flushright}} %13
\item $7x-3=9+4x$ \begin{flushright}\textcolor{red}{\framebox[8em]{\rule{0pt}{6ex}}\end{flushright}} %44
\item ある品物を仕入れて原価の40\%の利益を見込んで定価をつけたが売れなかったので、安売りの日に定価の20\%引きで売ったら480円の利益を得た。この品物の原価を求めなさい。 \vfill \begin{flushright}\textcolor{red}{\framebox[8em]{\rule{0pt}{6ex}}\end{flushright}} %29
\item 池の周りに道がある。太郎君と妹がこの周りを回った。同じ地点から同時にスタートし、反対方向に回ると10分で出会い、同じ方向に回ると30分で太郎君が妹に1周差をつけて追いついた。太郎君の速さが分速80mのとき妹の速さは分速何mか。 \vfill \begin{flushright}\textcolor{red}{\framebox[8em]{\rule{0pt}{6ex}}\end{flushright}} %17
\item aの9割を文字式で表せ \vfill \begin{flushright}\textcolor{red}{\framebox[8em]{\rule{0pt}{6ex}}\end{flushright}} %37
\end{enumerate}
    \vfill
\end{itemize}
\clearpage
 \begin{center}
   \LARGE\textbf{宮プリ ~一次方程式特訓編~\textcircled{\scriptsize 11}}
     \begin{flushright}
       名前\underline{\hspace{8zw}}
     \end{flushright}
 \end{center}

 \begin{itemize}
   \item 次の一次方程式をxについて解け
   \begin{enumerate}
\item $52x+1=34x$ \begin{flushright}\textcolor{red}{\framebox[8em]{\rule{0pt}{6ex}}\end{flushright}} %126
\item $\frac{x}{6} = \frac{3-x}{4}$ \begin{flushright}\textcolor{red}{\framebox[8em]{\rule{0pt}{6ex}}\end{flushright}} %140
\item $-3x=-27$ \begin{flushright}\textcolor{red}{\framebox[8em]{\rule{0pt}{6ex}}\end{flushright}} %7
\item $2+ \frac{3}{4} x= \frac{5}{6} x$ \begin{flushright}\textcolor{red}{\framebox[8em]{\rule{0pt}{6ex}}\end{flushright}} %112
\item $\frac{2}{3} x+ \frac{3}{4} = \frac{1}{2} - \frac{1}{2}x$ \begin{flushright}\textcolor{red}{\framebox[8em]{\rule{0pt}{6ex}}\end{flushright}} %134
\item $\frac{1}{2} x-1= \frac{1}{3}x+ \frac{5}{2}$ \begin{flushright}\textcolor{red}{\framebox[8em]{\rule{0pt}{6ex}}\end{flushright}} %159
\item $2x+15=-x$ \begin{flushright}\textcolor{red}{\framebox[8em]{\rule{0pt}{6ex}}\end{flushright}} %37
\item ある中学校の今年の生徒数は去年に比べて5\%増えて、441人でした。去年の生徒数を求めよ。 \vfill \begin{flushright}\textcolor{red}{\framebox[8em]{\rule{0pt}{6ex}}\end{flushright}} %41
\item 姉と弟がアメを持っている。姉は弟に比べて20個多く持っていた。姉が弟に自分のアメの 16 をあげたので二人のアメの数がちょうど同じになった。 アメは全部で何個あったか。 \vfill \begin{flushright}\textcolor{red}{\framebox[8em]{\rule{0pt}{6ex}}\end{flushright}} %26
\item aの3\%を文字式で表せ。 \vfill \begin{flushright}\textcolor{red}{\framebox[8em]{\rule{0pt}{6ex}}\end{flushright}} %36
\end{enumerate}
    \vfill
\end{itemize}
\clearpage
 \begin{center}
   \LARGE\textbf{宮プリ ~一次方程式特訓編~\textcircled{\scriptsize 11}}
     \begin{flushright}
       名前\underline{\hspace{8zw}}
     \end{flushright}
 \end{center}

 \begin{itemize}
   \item 次の一次方程式をxについて解け
   \begin{enumerate}
\item $14-15x=8-13x$ \begin{flushright}\textcolor{red}{\framebox[8em]{\rule{0pt}{6ex}}\end{flushright}} %55
\item $3(x-1)-2(x+3)=6-(2x+3)$ \begin{flushright}\textcolor{red}{\framebox[8em]{\rule{0pt}{6ex}}\end{flushright}} %163
\item $\frac{1}{2}x- \frac{1}{4} = \frac{3}{4} x$ \begin{flushright}\textcolor{red}{\framebox[8em]{\rule{0pt}{6ex}}\end{flushright}} %114
\item $3x=15$ \begin{flushright}\textcolor{red}{\framebox[8em]{\rule{0pt}{6ex}}\end{flushright}} %4
\item $3x=18-5x$ \begin{flushright}\textcolor{red}{\framebox[8em]{\rule{0pt}{6ex}}\end{flushright}} %39
\item $14+9x=-4$ \begin{flushright}\textcolor{red}{\framebox[8em]{\rule{0pt}{6ex}}\end{flushright}} %26
\item $3x-7=-2x+9$ \begin{flushright}\textcolor{red}{\framebox[8em]{\rule{0pt}{6ex}}\end{flushright}} %53
\item 花子と由美がおはじきを 30 個ずつ持っている。花子が由美に何個かあげたら、花子のおはじきが由美のおはじきの数のちょうど半分になった。花子は由美にいくつあげたのか。 \vfill \begin{flushright}\textcolor{red}{\framebox[8em]{\rule{0pt}{6ex}}\end{flushright}} %4
\item 8\%の食塩水がある。ここに3\%の食塩水を200g混ぜると6\%の食塩水になった。8\%の食塩水は何gあったのか。 \vfill \begin{flushright}\textcolor{red}{\framebox[8em]{\rule{0pt}{6ex}}\end{flushright}} %22
\item 1200mの道のりを一定の速さで40分で歩いた。このときの速さは分速何mか。 \vfill \begin{flushright}\textcolor{red}{\framebox[8em]{\rule{0pt}{6ex}}\end{flushright}} %12
\end{enumerate}
    \vfill
\end{itemize}
\clearpage
 \begin{center}
   \LARGE\textbf{宮プリ ~一次方程式特訓編~\textcircled{\scriptsize 11}}
     \begin{flushright}
       名前\underline{\hspace{8zw}}
     \end{flushright}
 \end{center}

 \begin{itemize}
   \item 次の一次方程式をxについて解け
   \begin{enumerate}
\item $x+3=-8$ \begin{flushright}\textcolor{red}{\framebox[8em]{\rule{0pt}{6ex}}\end{flushright}} %2
\item $\frac{7x+8}{12} = \frac{5x-9}{8}$ \begin{flushright}\textcolor{red}{\framebox[8em]{\rule{0pt}{6ex}}\end{flushright}} %148
\item $9-2(x-4)=3x+7$ \begin{flushright}\textcolor{red}{\framebox[8em]{\rule{0pt}{6ex}}\end{flushright}} %77
\item $\frac{4}{3} x+7=-2$ \begin{flushright}\textcolor{red}{\framebox[8em]{\rule{0pt}{6ex}}\end{flushright}} %31
\item $4x-7=8$ \begin{flushright}\textcolor{red}{\framebox[8em]{\rule{0pt}{6ex}}\end{flushright}} %29
\item $1.7x-0.7=1.6x-0.5$ \begin{flushright}\textcolor{red}{\framebox[8em]{\rule{0pt}{6ex}}\end{flushright}} %94
\item $x-4=7$ \begin{flushright}\textcolor{red}{\framebox[8em]{\rule{0pt}{6ex}}\end{flushright}} %1
\item 5\%の食塩水が何gかある。これに食塩を50g入れて、そのあと水を200g加え、さらに1\%の食塩水を400g加えてよくかき混ぜたら、6\%の食塩水ができた。5\%の食塩水は何gあったのか。 \vfill \begin{flushright}\textcolor{red}{\framebox[8em]{\rule{0pt}{6ex}}\end{flushright}} %25
\item ある商品を安売りのときに定価の2割引で売った。そのときの安売りの値段が720円だった。定価を求めよ。 \vfill \begin{flushright}\textcolor{red}{\framebox[8em]{\rule{0pt}{6ex}}\end{flushright}} %45
\item ある商品を2000円で仕入れた。この商品に定価をつけて、安売りのときに定価の3割引で売っても100円の利益が出るようにしたい。定価を求めよ。 \vfill \begin{flushright}\textcolor{red}{\framebox[8em]{\rule{0pt}{6ex}}\end{flushright}} %47
\end{enumerate}
    \vfill
\end{itemize}
\clearpage
 \begin{center}
   \LARGE\textbf{宮プリ ~一次方程式特訓編~\textcircled{\scriptsize 11}}
     \begin{flushright}
       名前\underline{\hspace{8zw}}
     \end{flushright}
 \end{center}

 \begin{itemize}
   \item 次の一次方程式をxについて解け
   \begin{enumerate}
\item $0.4x+0.6=0.2$ \begin{flushright}\textcolor{red}{\framebox[8em]{\rule{0pt}{6ex}}\end{flushright}} %90
\item $\frac{6}{5} x+ \frac{5}{2} =- \frac{3}{10} x-2$ \begin{flushright}\textcolor{red}{\framebox[8em]{\rule{0pt}{6ex}}\end{flushright}} %116
\item $11- \frac{4}{5} x= \frac{2}{3} x$ \begin{flushright}\textcolor{red}{\framebox[8em]{\rule{0pt}{6ex}}\end{flushright}} %113
\item $1.12-0.67x=0.06-1.2x$ \begin{flushright}\textcolor{red}{\framebox[8em]{\rule{0pt}{6ex}}\end{flushright}} %101
\item $4x+7=x-11$ \begin{flushright}\textcolor{red}{\framebox[8em]{\rule{0pt}{6ex}}\end{flushright}} %42
\item $\frac{2}{3} x+ \frac{2}{9} =2- \frac{2}{9} x$ \begin{flushright}\textcolor{red}{\framebox[8em]{\rule{0pt}{6ex}}\end{flushright}} %118
\item $2(3x-5)=3(x+2)+4(2x-1)$ \begin{flushright}\textcolor{red}{\framebox[8em]{\rule{0pt}{6ex}}\end{flushright}} %86
\item りんご 5 個とみかん 10 個買った。代金の合計は 1500 円だった。りんご 1 個の値段はみかん 1 個より 30 円高い。みかんは 1 個いくらか。 \vfill \begin{flushright}\textcolor{red}{\framebox[8em]{\rule{0pt}{6ex}}\end{flushright}} %0
\item 1200のt\%を文字式で表せ \vfill \begin{flushright}\textcolor{red}{\framebox[8em]{\rule{0pt}{6ex}}\end{flushright}} %34
\item 兄が 1800 円、弟が 1000 円持っていた。兄が鉛筆を4本買い、同じ鉛筆を弟が 3 本買った。すると兄の残金が弟の残金のちょうど 2 倍になった。鉛筆 1 本の値段を求めよ。 \vfill \begin{flushright}\textcolor{red}{\framebox[8em]{\rule{0pt}{6ex}}\end{flushright}} %6
\end{enumerate}
    \vfill
\end{itemize}
\clearpage
 \begin{center}
   \LARGE\textbf{宮プリ ~一次方程式特訓編~\textcircled{\scriptsize 11}}
     \begin{flushright}
       名前\underline{\hspace{8zw}}
     \end{flushright}
 \end{center}

 \begin{itemize}
   \item 次の一次方程式をxについて解け
   \begin{enumerate}
\item $2+3x=x+14$ \begin{flushright}\textcolor{red}{\framebox[8em]{\rule{0pt}{6ex}}\end{flushright}} %49
\item $3.3x+1.4=0.6x-1.3$ \begin{flushright}\textcolor{red}{\framebox[8em]{\rule{0pt}{6ex}}\end{flushright}} %98
\item $0.6x+0.4(3x-2)=4.6$ \begin{flushright}\textcolor{red}{\framebox[8em]{\rule{0pt}{6ex}}\end{flushright}} %105
\item $0.1x+0.3=1.8$ \begin{flushright}\textcolor{red}{\framebox[8em]{\rule{0pt}{6ex}}\end{flushright}} %157
\item $x-8=-3$ \begin{flushright}\textcolor{red}{\framebox[8em]{\rule{0pt}{6ex}}\end{flushright}} %3
\item $-3(4-x)=x$ \begin{flushright}\textcolor{red}{\framebox[8em]{\rule{0pt}{6ex}}\end{flushright}} %59
\item $5-x=4$ \begin{flushright}\textcolor{red}{\framebox[8em]{\rule{0pt}{6ex}}\end{flushright}} %16
\item 弟が家を出て毎分40mで歩く。その5分後に兄が毎分60mで追いかける。兄が弟に追いつくのは家から何mの地点か。 \vfill \begin{flushright}\textcolor{red}{\framebox[8em]{\rule{0pt}{6ex}}\end{flushright}} %14
\item 10\%の食塩水300gと1\%の食塩水を何gかをよく混ぜて、そこに食塩を20g入れ、さらにそこから水を70g蒸発させたら6\%の食塩水になった。1\%の食塩水は何gまぜたのだろうか。 \vfill \begin{flushright}\textcolor{red}{\framebox[8em]{\rule{0pt}{6ex}}\end{flushright}} %24
\item ある品物が定価の2割引で安売りしていた。その安売りの値段に消費税(5\%)がついて630円だった。この品物の定価を求めなさい。 \vfill \begin{flushright}\textcolor{red}{\framebox[8em]{\rule{0pt}{6ex}}\end{flushright}} %27
\end{enumerate}
    \vfill
\end{itemize}
\clearpage
 \begin{center}
   \LARGE\textbf{宮プリ ~一次方程式特訓編~\textcircled{\scriptsize 11}}
     \begin{flushright}
       名前\underline{\hspace{8zw}}
     \end{flushright}
 \end{center}

 \begin{itemize}
   \item 次の一次方程式をxについて解け
   \begin{enumerate}
\item $\frac{5x+4}{6} = \frac{3}{4}x+1$ \begin{flushright}\textcolor{red}{\framebox[8em]{\rule{0pt}{6ex}}\end{flushright}} %146
\item $1.2x+1.2=0.7x-1.3$ \begin{flushright}\textcolor{red}{\framebox[8em]{\rule{0pt}{6ex}}\end{flushright}} %164
\item $2x=x-11$ \begin{flushright}\textcolor{red}{\framebox[8em]{\rule{0pt}{6ex}}\end{flushright}} %34
\item $\frac{1}{2}(4x-8)=5x+2$ \begin{flushright}\textcolor{red}{\framebox[8em]{\rule{0pt}{6ex}}\end{flushright}} %84
\item $\frac{1}{2} x- \frac{4}{5} = \frac{4}{7} x+ \frac{3}{10}$ \begin{flushright}\textcolor{red}{\framebox[8em]{\rule{0pt}{6ex}}\end{flushright}} %135
\item $5-x=3x+13$ \begin{flushright}\textcolor{red}{\framebox[8em]{\rule{0pt}{6ex}}\end{flushright}} %43
\item $3x+2(5-x)=6$ \begin{flushright}\textcolor{red}{\framebox[8em]{\rule{0pt}{6ex}}\end{flushright}} %66
\item ある数と 5 との和の 3 倍はもとの数の 7 倍から 1 を引いたものと等しい。もとの数を求めよ。 \vfill \begin{flushright}\textcolor{red}{\framebox[8em]{\rule{0pt}{6ex}}\end{flushright}} %3
\item 80 円の鉛筆と 100 円のボールペンを合わせて 15 本買った。代金の合計は1340 円だった。それぞれ何本ずつ買ったのか。 \vfill \begin{flushright}\textcolor{red}{\framebox[8em]{\rule{0pt}{6ex}}\end{flushright}} %1
\item ある中学校の全生徒数は796人です。女子の人数は男子の人数の99\%です。この中学校の男子の人数と女子の人数をそれぞれ求めなさい。 \vfill \begin{flushright}\textcolor{red}{\framebox[8em]{\rule{0pt}{6ex}}\end{flushright}} %42
\end{enumerate}
    \vfill
\end{itemize}
\clearpage
 \begin{center}
   \LARGE\textbf{宮プリ ~一次方程式特訓編~\textcircled{\scriptsize 11}}
     \begin{flushright}
       名前\underline{\hspace{8zw}}
     \end{flushright}
 \end{center}

 \begin{itemize}
   \item 次の一次方程式をxについて解け
   \begin{enumerate}
\item $8x=2x+9$ \begin{flushright}\textcolor{red}{\framebox[8em]{\rule{0pt}{6ex}}\end{flushright}} %38
\item $\frac{3}{4} x-1=2$ \begin{flushright}\textcolor{red}{\framebox[8em]{\rule{0pt}{6ex}}\end{flushright}} %154
\item $7x+5=11$ \begin{flushright}\textcolor{red}{\framebox[8em]{\rule{0pt}{6ex}}\end{flushright}} %46
\item $7x-8=4(4+x)$ \begin{flushright}\textcolor{red}{\framebox[8em]{\rule{0pt}{6ex}}\end{flushright}} %65
\item $\frac{x}{8} -2= \frac{9-x}{6}$ \begin{flushright}\textcolor{red}{\framebox[8em]{\rule{0pt}{6ex}}\end{flushright}} %147
\item $- \frac{7}{9} x= \frac{28}{3}$ \begin{flushright}\textcolor{red}{\framebox[8em]{\rule{0pt}{6ex}}\end{flushright}} %107
\item $1.9x+0.8=4x-2.7$ \begin{flushright}\textcolor{red}{\framebox[8em]{\rule{0pt}{6ex}}\end{flushright}} %97
\item 200の15\%を求めよ \vfill \begin{flushright}\textcolor{red}{\framebox[8em]{\rule{0pt}{6ex}}\end{flushright}} %33
\item はじめ、兄の貯金額は弟の貯金額の 3 倍でした。二人とも毎月 1000 円ずつ貯金したら2ヵ月後には兄の貯金が弟の貯金の 2 倍になった。弟ははじめいくら貯金があったか。 \vfill \begin{flushright}\textcolor{red}{\framebox[8em]{\rule{0pt}{6ex}}\end{flushright}} %5
\item あるクラスの人数は女子が男子より5人多くて、男女合わせて37人です。このクラスの男子の人数を求めよ。 \vfill \begin{flushright}\textcolor{red}{\framebox[8em]{\rule{0pt}{6ex}}\end{flushright}} %2
\end{enumerate}
    \vfill
\end{itemize}
\clearpage
 \begin{center}
   \LARGE\textbf{宮プリ ~一次方程式特訓編~\textcircled{\scriptsize 11}}
     \begin{flushright}
       名前\underline{\hspace{8zw}}
     \end{flushright}
 \end{center}

 \begin{itemize}
   \item 次の一次方程式をxについて解け
   \begin{enumerate}
\item $- \frac{3}{4}x = 9$ \begin{flushright}\textcolor{red}{\framebox[8em]{\rule{0pt}{6ex}}\end{flushright}} %20
\item $4x+1=-11$ \begin{flushright}\textcolor{red}{\framebox[8em]{\rule{0pt}{6ex}}\end{flushright}} %22
\item $0.38-0.18x=0.2$ \begin{flushright}\textcolor{red}{\framebox[8em]{\rule{0pt}{6ex}}\end{flushright}} %93
\item $\frac{1}{2} x-  = \frac{5}{4} -x$ \begin{flushright}\textcolor{red}{\framebox[8em]{\rule{0pt}{6ex}}\end{flushright}} %130
\item $\frac{2}{3} (9-6x)-4(x+ \frac{1}{2} )=- \frac{3}{2} (8x-12)$ \begin{flushright}\textcolor{red}{\framebox[8em]{\rule{0pt}{6ex}}\end{flushright}} %88
\item $ 4-x= \frac{2x+7}{5}$ \begin{flushright}\textcolor{red}{\framebox[8em]{\rule{0pt}{6ex}}\end{flushright}} %143
\item $0.5(x-9)=-1.3x$ \begin{flushright}\textcolor{red}{\framebox[8em]{\rule{0pt}{6ex}}\end{flushright}} %102
\item 2400mの道のりを分速80mの一定の速さで歩いたら、かかる時間は何分か。 \vfill \begin{flushright}\textcolor{red}{\framebox[8em]{\rule{0pt}{6ex}}\end{flushright}} %13
\item 時速3kmで2時間歩いたときの道のりは何kmか。 \vfill \begin{flushright}\textcolor{red}{\framebox[8em]{\rule{0pt}{6ex}}\end{flushright}} %11
\item ある中学校では全校生徒の45\%が女子である。男子の人数は女子の人数より24人多い。この学校の全校生徒の人数を求めよ。 \vfill \begin{flushright}\textcolor{red}{\framebox[8em]{\rule{0pt}{6ex}}\end{flushright}} %43
\end{enumerate}
    \vfill
\end{itemize}
\clearpage
 \begin{center}
   \LARGE\textbf{宮プリ ~一次方程式特訓編~\textcircled{\scriptsize 11}}
     \begin{flushright}
       名前\underline{\hspace{8zw}}
     \end{flushright}
 \end{center}

 \begin{itemize}
   \item 次の一次方程式をxについて解け
   \begin{enumerate}
\item $23x=-12x-1$ \begin{flushright}\textcolor{red}{\framebox[8em]{\rule{0pt}{6ex}}\end{flushright}} %127
\item $ 2x-1= \frac{5x+11}{8}$ \begin{flushright}\textcolor{red}{\framebox[8em]{\rule{0pt}{6ex}}\end{flushright}} %145
\item $5+3x=4$ \begin{flushright}\textcolor{red}{\framebox[8em]{\rule{0pt}{6ex}}\end{flushright}} %47
\item $-4x+5=-11$ \begin{flushright}\textcolor{red}{\framebox[8em]{\rule{0pt}{6ex}}\end{flushright}} %25
\item $6(1-x)=-x+11$ \begin{flushright}\textcolor{red}{\framebox[8em]{\rule{0pt}{6ex}}\end{flushright}} %62
\item $1.1x+0.6=0.85x-1.15$ \begin{flushright}\textcolor{red}{\framebox[8em]{\rule{0pt}{6ex}}\end{flushright}} %99
\item $13+4x=25$ \begin{flushright}\textcolor{red}{\framebox[8em]{\rule{0pt}{6ex}}\end{flushright}} %27
\item 濃度のわからない食塩水が200gある。ここに10\%の食塩水を300g混ぜたら8\%の食塩水ができた。はじめにあった食塩水の濃度は何\%だったか。 \vfill \begin{flushright}\textcolor{red}{\framebox[8em]{\rule{0pt}{6ex}}\end{flushright}} %21
\item ある日、A君は家から11.5km離れた野球場に行った。10時に家を出てバス停まで歩き、そのバス停で5分間待ってからバスに乗り、野球場近くのバス停で降りて野球場まで歩いたら10時50分に着いた。A君の歩く速さは毎分60m,バスの速さは毎時30kmでそれぞれ一定だったとする。A君がバスに乗っていたのは何分間だったか求めよ。 \vfill \begin{flushright}\textcolor{red}{\framebox[8em]{\rule{0pt}{6ex}}\end{flushright}} %20
\item 50の3割を求めよ \vfill \begin{flushright}\textcolor{red}{\framebox[8em]{\rule{0pt}{6ex}}\end{flushright}} %32
\end{enumerate}
    \vfill
\end{itemize}
\clearpage
\end{document}
\end{document}
